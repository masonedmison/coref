
\documentclass[11pt]{article}
\usepackage[margin = 1in]{geometry}
\usepackage[none]{hyphenat}
\usepackage{fancyhdr}
\usepackage{graphicx}
\usepackage{float}





\pagestyle{fancy}
\fancyhead{}
\fancyfoot{}
\fancyhead[L]{\slshape \MakeUppercase{Term Project Proposal}}
\fancyhead[R]{\slshape Mason Edmison}
\fancyfoot[C]{\thepage}

%%%%
% hack to remove indent
\newlength\tindent
\setlength{\tindent}{\parindent}
\setlength{\parindent}{0pt}
\renewcommand{\indent}{\hspace*{\tindent}}
%%%%

\begin{document}

\begin{titlepage}
\begin{center}
\Large{\textbf{Term Project Proposal}} \\
\Large{\textbf{CS 710 - Artificial Intelligence}} \\

\vfill
\line(1,0){400} \\

\Large{\textbf{Coreference Resolution in Biomedical Text:}} \\
\Large{\textbf{Improving Performance with Ontologies and Domain Specific NER}} \\

\line(1,0){400}\\
\vfill
Mason Edmison\\
University of Wisconsin-Milwaukee\\
10/25/2019
\end{center}
\end{titlepage}

\section{Introduction}


\section{Dataset and Evaluation}
Due to limited resources and time, I will not be able to train the coreference systems used and will use the pre-trained models\footnote{Both systems, \emph{deep coref} and \emph{neural coref} ship pre-trained on the CoNLL 2012 dataset}. For evaluating the  coref system, I will use the training dataset from the task Protein Coreference at BioNLP 2011. I will evaluate the system on both mention detection as well as mention linking to produce coreference links using the harmonic mean \emph{F1} as the determining metric. 

\section{Performance}
\begin{center}
 \begin{tabular}{||c c c c||} 
 \hline
  &  Prec.  & Rec. & $F1$ \\ [0.5ex] 
 \hline\hline
     SpaCy Web Md & 15.15\% & 8.37\% & 10.72\% \\ 
 \hline
\end{tabular}
\end{center}

\newpage
\bibliographystyle{plain}
\bibliography{References}

\nocite{pilehvar-collier-2016-improved, choi-etal-2014-analysis, Prokofyev:2015:SOC:2942298.2942337, 95e005df001d47be8468f24479845109,clark-manning-2016-improving}

\end{document} % end document
